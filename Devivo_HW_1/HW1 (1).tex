\documentclass{article}
\usepackage[cm,plain]{fullpage}
\usepackage{enumerate}           % This package gives fancier enumeration styles
\usepackage{times}               % This package provides use of Times font


%==============================================================
% EDIT: Change your name and the assignment number if necessary.
\author{John Devivo}          % Use your name instead
\title{Homework \#1}       % Note the \ before the # symbol 
                           % # is a special character

%==============================================================
% This is where the real document begins
% 0.1 (a-d), 0.2 (a-d)
% 1.3, 1.4 (a,c,e), 1.5 (c,g), 1.6 (a-d)

\begin{document}
\maketitle
\begin{center}     % Start a centered block of text
\Large{\bf Due: Tuesday, September 9}\\
Collaborators: Heather Corracio, Ryan Schwarz
\end{center}       % Centering ends here

\section*{E-0.1}    % Section* creates a new section without a section number
Examine the following formal descriptions of sets so that you understand which
members they contain.
Write a short informal English description of each set.
\begin{enumerate}[a.]
\item $\{1, 3, 5, 7, \dots \}$\\
{\bf Answer: } The set of all numbers that can be expressed as the sum of one and two times a nonnegative integer n.

\item $\{\dots, -4, -2, 0, 2, 4, \dots \}$\\
{\bf Answer: } The set of all numbers that can be expressed as the product of two and an integer n.

\item $\{n | n = 2m \mbox{\ for some $m$ in $\cal{N}$}\}$\\
{\bf Answer: } The set of all numbers n, where n can be expressed as the product of two and a number m, where m is an element of set N.

\item $\{n | n = 2m \mbox{\ for some $m$ in $\cal{N}$, and $n = 3k$ for some $k$ in $\cal{N}$}\}$\\
{\bf Answer: } The set of all numbers n, where n can be expressed as BOTH two times a number m, where m is an element of set N, AND three times a number k, where k is an element of set N.

\end{enumerate}

\section*{E-0.2}
Write formal descriptions of the following sets:
\begin{enumerate}[a.]
\item The set containing the number 1, 10, and 100\\
{\bf Answer: } $\{1,10,100\}$
% Look at previous question for tips on typesetting sets in math mode

\item The set containing all integers that are greater than 5\\
{\bf Answer: } $\{6,7,8,\dots\}$

\item The set containing all natural numbers that are less than 5\\
{\bf Answer: } $\{1,2,3,4\}$

\item The set containing the string \verb=aba=\\
{\bf Answer: } $\{aba\}$

\end{enumerate}

\section*{E-1.3}

The formal description of a DFA $M$ is $(\{q_1, q_2, q_3, q_4, q_5\}, \{{\tt u}, {\tt d}\},
\delta, q_3, \{q_3\})$,
where $\delta$ is given by the following table.
Give the state diagram of this machine.
\begin{center}
$\begin{array}{c|cc}
     & {\tt u} & {\tt d}\\
\hline
q_1  & q_1     & q_2\\
q_2  & q_1     & q_3\\
q_3  & q_2     & q_4\\
q_4  & q_3     & q_5\\
q_5  & q_4     & q_5
\end{array}
$
\end{center}

{\bf Answer: } Save this as the file \verb=1_3.jff=
% Note the \verb=...= sets everything in plain ASCII (verbatim) text
% until the matching = symbol appears.
% So, the file should be called: 1_3.jff

\section*{E-1.4}
% a,c,e
Each of the following languages is the intersection of two simpler languages.
In each part, construct DFAs for the simpler languages, then combine them using
the construction discussed in footnote 3 (page 46) to give the state diagram
of a DFA for the language given.
In all parts, $\Sigma = \{a, b\}$.
\begin{enumerate}[a.]
\item $\{ w | w \mbox{\ has at least three {\tt a}'s and at least two {\tt b}'s}\}$
\stepcounter{enumi}
\item $\{ w | w \mbox{\ has an even number of {\tt a}'s and one or two {\tt b}'s}\}$
\stepcounter{enumi}
\item $\{ w | w \mbox{\ starts with an {\tt a} and has at most one {\tt b}}\}$
\end{enumerate}

{\bf Answer: } You only have to show the final diagram for each problem.
(If you are having difficulty combining, then save and submit the 
intermediate diagrams as well.)
Save these as the files \verb=1_4_a.jff, 1_4_c.jff, 1_4_e.jff=
% Again, the files are: 1_4_a.jff, 1_4_c.jff, and 1_4_f.jff

\section*{E-1.5}
Each of the following languages is the complement of a simpler language.
In each part, construct a DFA for the simpler language, then use it to give the
state diagram of a DFA for the language given.
In all parts, $\Sigma = \{a, b\}$.
\begin{enumerate}[a.]
\setcounter{enumi}{2}
\item $\{w | w \mbox{\ contains neither the substrings {\tt ab} nor {\tt ba}}\}$
\setcounter{enumi}{6}
\item $\{w | w \mbox{\ is any string that doesn't contain exactly two {\tt a}'s}\}$
\end{enumerate}

{\bf Answer: } You only have to show the final diagram for each problem.
Save these as the files \verb=1_5_c.jff, 1_5_g.jff=

\section*{E-1.6}
Give state diagrams of DFAs recognizing the following languages.
In all parts, the alphabet is $\{0,1\}$.
\begin{enumerate}[a.]
\item $\{ w | w \mbox{\ begins with a 1 and ends with a 0}\}$
\item $\{ w | w \mbox{\ contains at least three 1s}\}$
\item $\{ w | w \mbox{\ contains the substring 0101 (i.e., $w = x0101y$ for some
$x$ and $y$)}\}$
\item $\{ w | w \mbox{\ has length at least 3 and its third symbol is a 0}\}$
\end{enumerate}

{\bf Answer: } Save these as the files \verb=1_6_a.jff, 1_6_b.jff, 1_6_c.jff, 1_6_d.jff=

\end{document}
