\documentclass{article}
\usepackage[cm,plain]{fullpage}
\usepackage{enumerate}           % This package gives fancier enumeration styles
\usepackage{times}               % This package provides use of Times font
\usepackage{graphicx}            % This package provides use of graphics


%==============================================================
% EDIT: Change your name and the assignment number if necessary.
\author{John Devivo}          % Use your name instead
\title{Homework \#4}       % Note the \ before the # symbol 
                           % # is a special character

%==============================================================
% This is where the real document begins
% 2.1.c, 2.2, 2.4.e, 2.6.b, 2.19

\begin{document}
\maketitle
\begin{center}     % Start a centered block of text
\Large{\bf Due: Thursday, October 2}\\
Collaborators: None
\end{center}       % Centering ends here

\section*{E-2.1} %c
Recall the CFG $G_4$ that we gave in Example 2.4. 
For convenience, let's rename the variables with single letters as follows.
\begin{eqnarray*}
E & \rightarrow & E\ +\ T\ |\ T\\
T & \rightarrow & T\ \times\ F\ |\ F\\
F & \rightarrow & (E)\ |\ \verb=a=
\end{eqnarray*}

Give parse trees and derivations for each of the following (just c):
\begin{enumerate}[a.]
\setcounter{enumi}{2}
\item \verb=a+a+a=\\
{\bf Answer: } % Your answer goes here...
\begin{verbatim}
            E
          /  \
         E  + T
        / \   |
       E + T  F
       |   |  |
       T   F  a
       |   |
       F   a
       |
       a
\end{verbatim}
Here is a partial derivation:\\

$E\rightarrow \\
E + T\rightarrow \\
E + F\rightarrow \\
E + a\rightarrow \\
E + T + a\rightarrow \\
E + F + a\rightarrow \\
E + a + a\rightarrow \\
T + a + a\rightarrow \\
F + a + a\rightarrow \\
a + a + a
$
\end{enumerate}

\section*{E-2.2}
\begin{enumerate}[a.]
\item Use the languages $A=\{a^mb^nc^n|\ m,n \geq 0\}$ and
$B=\{a^nb^nc^m|\ m,n \geq 0\}$ together with Example 2.36 to show
that the class of context-free languages is not closed under intersection.\\
{\bf Answer: } We begin by assuming that Context Free Languages are closed under the intersection operation for the sake of contradiction. \\
The first step in using languages $A$ and $B$ to show that CFLs are closed under intersection is to prove that both languages are CFLs. We can do this by constructing  context free grammars to represent $A$ and $B$.\\
A context free grammar that represents $A$ can be shown as \\ \\
$S\rightarrow FG | F | G | \epsilon$ \\
$G\rightarrow bc | bGc | \epsilon $ \\
$F\rightarrow a | aF | \epsilon$ \\ \\
thus $A$ is a context free language. \\ \\
A context free grammar that represents $B$ can be shown as \\ \\
$S\rightarrow FG | F | G | \epsilon$ \\
$F\rightarrow ab | aFb | \epsilon$ \\
$G\rightarrow c | cG | \epsilon$ \\ \\

thus $B$ is a context free language. \\ \\

If the class of context free languages were closed under intersection, the intersection of $A$ and $B$, $A \cap B = C = \{a^nb^nc^n | n \geq 0\}$ would also be a context free language. However, example 2.36 in the book uses the pumping for context free languages to prove that $C$ is not a context free language. We have a contradiction, context free languages are not closed under intersection.

\item Use part (a) and DeMorgan's law (Theorem 0.20) to show that the class
of context-free languages is not closed under complementation.\\
{\bf Answer: } 
In order to prove that the set of context free languages is not closed under compliment, we first assume that they are closed under compliment. This would imply that $D =  \bar{A}$ is a context free grammar and $E = \bar{B}$ is a context free language. Since Context free languages are closed under the union operation, we know that $F = D\cup E$ is a context free language. \\ \\
If CFLs are closed under compliment, we know that $\bar{F}$ is a context free language. Using DeMorgan's law we know that $F = C$, however we showed in part a that $C$ is not a context free language, thus we have our contradiction. Context Free Languages are not closed under compliment.
\end{enumerate}

\section*{E-2.4} %e
Give context-free languages that generate the following languages.
In all parts, the alphabet is $\Sigma = \{0,1\}$.
\begin{enumerate}[a.]
\setcounter{enumi}{4}
\item $\{w = w^R, \mbox{\ that is, $w$ is a palindrome.}\}$\\
{\bf Answer: } $S\rightarrow0S0|1S1|=$
\end{enumerate}

\section*{E-2.6} %b
Give context-free grammars generating the following languages.
\begin{enumerate}[a.]
\setcounter{enumi}{1}
\item The complement of the language $\{a^nb^n|n\geq 0\}$.\\
{\bf Answer: }
$S\rightarrow A | B$ \\
$A\rightarrow 1A0 | 1A | 1$ \\
$B\rightarrow 1B0 | B0 | 0$
\end{enumerate}

\section*{E-2.19}
Let CFG $G$ be the following grammar.
\begin{eqnarray*}
S & \rightarrow & aSb\ |\ bY\ |\ Ya\\
Y & \rightarrow & bY\ |\ aY\ |\ \epsilon
\end{eqnarray*}
Give a simple description of $L(G)$ in English.
Use that description to give a CFG for $\overline{L(G)}$, the complement of $L(G)$.\\
{\bf Answer: } The CFG $G$ describes the language of all permutations of $n \geq 0 a$ characters and $m \geq 0 b$ characters but also excludes the empty set. The compliment of $L(G)$ would be a language that only describes the empty string $\{ \epsilon \}$.

\end{document}
